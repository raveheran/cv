%%%%%%%%%%%%%%%%%%%%%%%%%%%%%%%%%%%%%%%%%
% Medium Length Professional CV
% LaTeX Template
% Version 2.0 (8/5/13)
%
% This template has been downloaded from:
% http://www.LaTeXTemplates.com
%
% Original author:
% Trey Hunner (http://www.treyhunner.com/)
%
% Important note:
% This template requires the resume.cls file to be in the same directory as the
% .tex file. The resume.cls file provides the resume style used for structuring the
% document.
%
%%%%%%%%%%%%%%%%%%%%%%%%%%%%%%%%%%%%%%%%%

\documentclass{resume} % Use the custom resume.cls class

\usepackage[utf8]{inputenc}
\RequirePackage[%
colorlinks,
linkcolor=black,
citecolor=black,
urlcolor=blue,
pdfusetitle
]{hyperref}

\usepackage[left=0.75in,top=0.2in,right=0.75in,bottom=0.6in]{geometry}

% citations
\usepackage[natbib, citestyle=numeric, bibstyle=ieee, maxbibnames=999, maxcitenames=2, mincitenames=1]{biblatex}
\addbibresource{mybib.bib}

% ADD PICTURE

\name{Eran Raveh}
%\address{123 Broadway \\ City, State 12345} % Your address
%\address{\url{ravehe@gmail.com}}
\address{\url{ravehe@gmail.com}}
\address{\url{http://www.coli.uni-saarland.de/~raveh/}}
%\address{+49 (0)17 310 416 99\quad$\vert$\quad+972 (0)50 218 7899}

\begin{document}

\begin{rSection}{Education}

\begin{rSubsection}
	{Ph.D.\ in Language Science and Technology}
	{\textsc{Apr} 2016 -- Present (expected Q4 2020)}
	{Universität des Saarlandes}
	{Saarbrücken, Germany} %(with )}
	
	\setlength{\itemindent}{.7cm}
		
	\item \textbf{Dissertation}: Modeling and Integrating Vocal Accommodation in Human-Computer  \\\hspace*{3.35cm}Interaction into Spoken Dialogue Systems %\\\hspace*{.7cm}(supervisor: Dr.\ Ingmar Steiner)
	\item Integrated M.Sc.\ in Speech Technology
\end{rSubsection}

\begin{rSubsection}
	{Visiting student}
	{2014}
	{Trinity College Dublin, Computer Science and Communication Science faculties}
	{Dublin, Ireland}
	
	\setlength{\itemindent}{.7cm}
	
	\item[]
	%\item \textbf{Courses}: Formalisms of Linguistics Theories, Information Management, Written Language
\end{rSubsection}

\begin{rSubsection}
	{B.A.\ in International Studies in Computational Linguistics}
	{2012 -- 2015}
	{Eberhard Karls Universität Tübingen}
	{Tübingen, Germany}
	
	\vspace*{-.2cm}
	\item[]{final grade: 93 {\footnotesize (US: A; EU ECTS: 1,4; UK: first-class honours)}}
	\vspace*{.2cm}

	\setlength{\itemindent}{.7cm}
	
	\item \textbf{Thesis}: Automatic Generation of Customizable Recall Questions
	%\\\hspace*{.7cm}(supervisor: Prof.\ Dr.\ Detmar Meurers)
	\item \textbf{Projects}: Continuous lexical formality analysis, stimuli discrimination in speech, music \& language
	%\item \textbf{Seminars}: NLP Tools Development, Distributional Semantics, Intelligent Tutoring Systems, \\\hspace*{.7cm}Text Simplification, Grammar Formalisms, Statistical NLP, and more
\end{rSubsection}

\end{rSection}

\begin{rSection}{Experience}

\begin{rSubsection}
	{Speech Scientist Intern}
	{\textsc{Jan} 2020 -- \textsc{Apr} 2020}
	{Fraunhofer IIS}
	{Erlangen, Germany}
	
	\setlength{\itemindent}{.7cm}
	
	\item \textbf{Research project}: implementing and evaluating phonemic representations for flexible\\\hspace*{4.25cm}training of multilingual Tacotron-based speech synthesis
	\item Development work on neural text-to-speech components of a voice assistant
\end{rSubsection}

\begin{rSubsection}
	{Researcher}
	{\textsc{Apr} 2016 -- Present}
	{Multimodal Computing and Interaction}
	{Saarbrücken, Germany}
	
	\setlength{\itemindent}{.7cm}
	
	\item Co-Researcher in the project \textit{Phonetic Convergence in Human-Computer Interaction} as\\\hspace*{.7cm}part of the independent research group \textit{Multimodal Speech Processing}
	\item Developing spoken dialogue systems and human-computer interaction applications for\\\hspace*{.7cm}demonstrations and proof-of-concept research
\end{rSubsection}

\begin{rSubsection}
	{Research Assistant}
	{\textsc{Oct} 2015 -- \textsc{Mar} 2016}
	{Universität Stuttgart}
	{Stuttgart, Germany}
	
	\setlength{\itemindent}{.7cm}
	
	\item Building domain-specific language models for multilingual automatic speech recognition
	\item Collecting corpora for speech applications using deep-structure web text scraping
\end{rSubsection}

\begin{rSubsection}
	{Language Engineer Intern}
	{\textsc{Oct} 2014 -- \textsc{Mar} 2015}
	{Voysis}
	{Dublin, Ireland}
	
	\setlength{\itemindent}{.7cm}

	\item Creating NLP resources and algorithms for multilingual speech synthesis
	\item Developing analysis and visualization processes for text-to-speech platforms
\end{rSubsection}

\begin{rSubsection}
	{Research Assistant and Teaching Assistant}
	{2013 -- 2015}
	{Eberhard Karls Universität Tübingen}
	{Tübingen, Germany}
	
	\setlength{\itemindent}{.7cm}
	
	\item Implementing and evaluating tools used for NLP research %(by Prof.\ Christopher Culy)
	\item Co-teaching courses in Syntax, Semantics, Phonology and Phonetics %(directed by Dr.\ Christian Ebert)
\end{rSubsection}

\begin{rSubsection}
	{Automation Engineer}
	{2010 -- 2012}
	{GE Healthcare}
	{Haifa, Israel}
	
	\setlength{\itemindent}{.7cm}
	
	\item Developing automation processes for software testing and specifications writing
	\item Contact person for national and global teams 
\end{rSubsection}

\end{rSection}

% \pagebreak

\begin{rSection}{Community}
	\begin{tabular}{l}
		YRRSDS (SIGdial satellite, \url{http://yrrsds.org}): chair (2018-2019), co-local organizer (2017)\\
		%Committee chair & YRRSDS 2018, Melbourne, Australia\\
		%Co-local organizer  & YRRSDS 2017, Saarbrücken, Germany
		Reviewer/sub-reviewer: COLING 2020, SpeCom 2020, LREC 2020, Interspeech 2019
	\end{tabular}
\end{rSection}

\begin{rSection}{Skills and Qualifications}

\textbf{ISTQB Software Testing Certification} from ITCB Foundation

\begin{tabular}{ @{} >{\bfseries}l @{\hspace{6ex}} l }
	Programming Languages 	& \underline{Proficient}: Python, Java, R\\
						  	& \underline{Familiar with}: Groovy, JavaScript, SQL, VB, DOS-batch\\[0.2cm]
	Tools \& Frameworks   	& PyTorch, Praat, OpenDial, CoreNLP\\
							& Docker, Gradle, Git, \LaTeX\\
\end{tabular}

\end{rSection}

\begin{rSection}{Languages}
	
	\begin{tabular}{ @{} >{\bfseries}l @{\hspace{3ex}} l @{\hspace{6ex}} @{} >{\bfseries}l @{\hspace{3ex}} l @{\hspace{6ex}}  @{} >{\bfseries}l @{\hspace{3ex}} l}
		Hebrew	& 	Native 		        &   English		& Fluent		&	German	& Advanced (C1) \\
		Arabic	&	Intermediate (B1)	&	Japanese	& Beginner (A2)
	\end{tabular}
	
\end{rSection}

%\begin{rSection}{Journals}
%
%	\nocite{Raveh2019journal}
%	
%	\section*{} % only ncessary to make bib numbers show
%	\printbibliography[heading=none]
%
%\end{rSection}

\begin{rSection}{Selected peer-reviewed publications}
	
	\newrefsection
	
	\nocite{Gessinger2020Interspeech}
%	\nocite{Raveh2020Specom}
	\nocite{Raveh2020Fraunhofer}
	\nocite{Cohn2020Interspeech}
%	\nocite{Gessinger2020Interspeech}
	\nocite{Raveh2020SpeechProsody}
	\nocite{Raveh2019InterspeechAlexa}
%	\nocite{Gessinger2019Interspeech}
	\nocite{Raveh2019SPCC}
%	\nocite{Raveh2019PundP}
	\nocite{Gessinger2019ICPhS}
	\nocite{Raveh2019ESSV}
	\nocite{Raveh2018Specom}
%	\nocite{Gessinger2018PuP}
%	\nocite{Gessinger2018SpeechProsody}
	\nocite{Jonel2018LREC}
%	\nocite{Raveh2017PundP}
%	\nocite{Raveh2017ISCOL}
	\nocite{Raveh2017SemDial}
	\nocite{Raveh2017Interspeech}
%	\nocite{Gessinger2017Interspeech}
	\nocite{Raveh2017ESSV}
%	\nocite{Gessinger2016PundP}
	
	\section*{} % only ncessary to make bib numbers show
	\printbibliography[heading=none]
	
%	\renewcommand{\section}[2]{}%	
%	\bibliographystyle{IEEEtran}
%	\bibliography{mybib}
\end{rSection}

\end{document}
